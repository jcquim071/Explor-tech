\documentclass[
    a4paper, 
    12pt, onecolumn,
    %draft
]{article}

%%% Pour un texte en francais
\usepackage[francais]{babel}
%\usepackage[latin1]{inputenc}	         % encodage des lettres accentuees
\usepackage[utf8]{inputenc}          % encodage des lettres accentuees
%\usepackage{graphicx}
\usepackage{graphicx} \def\BIB{}

\usepackage{mathptmx}



%%% Quelques raccourcis pour la mise en page
\newcommand{\remarque}[1]{{\small \it #1}}
\newcommand{\rubrique}{\bigskip \noindent $\bullet$ }

\begin{document}

\noindent EXPLOR \hfill \textsc{Demande d'Attribution de Ressources Informatiques}

\begin{center}
\Large  \bf
Description scientifique du projet
\end{center}
\bigskip

\rubrique  Titre du projet:  
\hfill
%%% METTRE ICI LE RENSEIGNEMENT DEMANDE
[TITRE DU PROJET]



\rubrique  Num\'ero du projet EXPLOR
\footnote{ Uniquement en cas de demande de prolongation d'un projet
  existant.} : 
\hfill
%%% METTRE ICI LE RENSEIGNEMENTS DEMANDE
[Num\'ero de projet]

\rubrique  Responsable scientifique
\footnote{ Le responsable se charge du suivi du projet et fournit un bilan
  en fin d'ann\'ee.} :  
\hfill
%%% METTRE ICI LES RENSEIGNEMENTS DEMANDES
[NOM Pr\'enom] 


\rubrique Laboratoire:  
\hfill
%%% METTRE ICI LES RENSEIGNEMENTS DEMANDES
[Universit\'e ou Institution - Nom du labo] 


%\newpage
\section{Contexte scientifique et objectifs du projet}

%%% A COMMMENTER LORS DE LA REDACTION DU PROJET
\emph{Longueur typique {\bf 1 page}, longueur maximale de {\bf 2 pages}. Si le projet se d\'ecompose en sous-projets, {\bf 1 page additionnelle maximum par sous-projet}.}
\vskip 0.2cm  

\section{M\'ethode numérique \& logiciel(s) employé(s)}

\emph{Longueur typique de  {\bf 0.5 page}, longueur maximale de {\bf 2 pages}.
La justification d'un accès aux ressources informatiques d'EXPLOR doit être argumentée.}


%\newpage
\section{Bibliographie}
\label{Sec:Biblio}
\emph{Bibliographie de l'équipe sur le projet}

\bibliographystyle{plain}
\bibliography{mybiblio}

\end{document}
%%%%%%%%%%%%%%%%%  Fin du fichier Latex  %%%%%%%%%%%%%%%%%%%%%%%%%%%%%%

